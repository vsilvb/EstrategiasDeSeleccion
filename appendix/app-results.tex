%!TEX root = ../main/main.tex

\subsection{Proof of Proposition 1 ~\ref{prop:myproposition}}


\begin{proof}
	We show two inclusions.
	
	\medskip\noindent\emph{(1) 	\(\llbracket E_{\max}\rrbracket_{d}
		\subseteq \textsc{Max}(\llbracket E_{\det}\rrbracket_{d})\)} 
		
	If \(E_{\max}\) accepts via a run \(\rho_{\max}\) producing mapping
	\(\mu\), then \(\mu\) is maximal under \(\preceq_{\mathrm{varInc}}\).
	
	Indeed, acceptance in \(E_{\max}\) means that in the final state
	\((R,W)\in F_{\max}\) we have
	\(R\cap F_{\det}\neq\emptyset\) (so \(\rho_{\max}\) yields \(\mu\))
	and \(W\cap F_{\det}=\emptyset\).  The fact
	\(W\cap F_{\det}=\emptyset\) guarantees that no “witness” run that
	dominates the reference run ever reached an accepting configuration.
	But any mapping \(\nu\succ_{\mathrm{varInc}}\mu\) in
	\(\llbracket E_{\det}\rrbracket_{d}\) would correspond to some such
	dominating run in the subset–construction, and hence would force
	\(W\cap F_{\det}\neq\emptyset\).  Contradiction.  Therefore \(\mu\)
	cannot be strictly extended, i.e.\ it is maximal.
	
	\medskip\noindent\emph{(2) 	\( \textsc{Max}(\llbracket E_{\det}\rrbracket_{d}) \subseteq \llbracket E_{\max}\rrbracket_{d}
		\)}
		
	If $\mu$ is maximal in \(\llbracket E_{\det}\rrbracket_{d}\) , $E_{max} $ accepts $\mu$. 	Let \(\rho\) be the unique accepting run of \(E_{\det}\) that
	produces \(\mu\);
	denote by
	\(X_{0}\xrightarrow{(a_{1},S_{1})}X_{1}\dots
	X_{n}\) the sequence of subset–states visited by~\(\rho\).
	We simulate \(\rho\) in \(E_{\max}\) and show inductively that after
	processing the first \(k\) input positions (\(0\le k\le n\))
	the witness component \(W_{k}\) contains \emph{no} final states:
	\[
	W_{k}\cap F_{\det} = \varnothing, 
	\qquad
	 X_{k} \in R_{k}.
	\]
	
	\begin{itemize}
		\item \emph{Base case.}  Initially \((R,W)=(X_{0},\emptyset)\);
		clearly \(W\cap F_{\det}=\emptyset\).
		
		\item \emph{Inductive step.}  Assume the invariant for \(k\).
		When reading \((a_{k+1},S_{k+1})\) the reference component follows
		\(\rho\) to \(X_{k+1}\).
		The update rule adds to \(W\)
		(a)~successors of the existing witnesses;
		(b)~any successor of \(X_{k}\) that emits a
		\emph{strict superset} of \(S_{k+1}\).
		If any state inserted under~(b) could reach \(F_{\det}\), then the
		corresponding run would realise a mapping
		\(\nu\succ_{\mathrm{varInc}}\mu\),
		contradicting the maximality of \(\mu\).
		By induction the invariant holds for \(k+1\).
		
		.
	\end{itemize}
	
	After the last symbol we have
	\(R_{n}\cap F_{\det}\neq\varnothing\) (contains \(X_{n}\))
	and \(W_{n}\cap F_{\det}=\varnothing\), so the final composite state
	is accepting.  Hence \(E_{\max}\) outputs \(\mu\).
	
	\medskip
	Combining (1) and (2) yields the desired equality
	\(\llbracket E_{\max}\rrbracket_{d} = \textsc{Max}(
	\llbracket E_{\det}\rrbracket_{d})\).
\end{proof}
